%
% Project proposal
% Guidelines can be found at: http://courses.cse.tamu.edu/caverlee/csce470/project.html
% 
\documentclass{article}
\renewcommand{\contentsname}{Table of Contents}
\newcommand{\PaperTitle}{Checkpoint}
\usepackage{hyperref}
%\usepackage{fullpage}

\begin{document}

%
% Title Page
\newcommand{\HRule}{\rule{\linewidth}{0.5mm}}
%
% Title Page
%
\begin{titlepage}
  \begin{center}
  % Title
  \textsc{\Large FootTraffic}\\[0.5cm]
  \HRule \\[0.4cm]
  {\huge \bfseries \PaperTitle}\\[0.2cm]
  \HRule \\[1cm]
  
  \large{Department of Computer Science \& Engineering} \\
  \large{Texas A\&M University} \\[1cm]

  % Authors and emails
  \begin{minipage}{0.4\textwidth}
    \begin{flushleft} \large
    \emph{Authors:}\\
    William \textsc{Chen} \\
    Eric \textsc{Wood}
    \end{flushleft}
  \end{minipage}
  \begin{minipage}{0.4\textwidth}
    \begin{flushright} \large
    \emph{Emails:} \\
    wchen16@cse.tamu.edu\\
    ebw0178@cse.tamu.edu
    \end{flushright}
  \end{minipage}
  \\[1cm]
  \large{\today}
  \vfill
  
  % Bottom of the page
  \end{center}
\end{titlepage}


%
% Document
\section{Milestone Walk}
We set out a base set of technologies and resources we want to use, and in this milestone, we installed,
configured, and learned them more in-depth so we can use them appropriately. We were off on a slow start,
and that was expected. We didn't cover much ground code-wise, but we do have the environment set up. We have
requested an edu account on GitHub for private repos, set up a production site on Heroku, toyed with 
Amazon EC2, looked into Google, Yelp, and Twitter APIs, considered (considering) Apache Lucene,
and installing and gaining traction with Ruby on Rails. 

\section{Milestone Jog}
During this milestone, we planned out our backend strategy and briefly brainstormed how the UI looks like and 
what features we should include/concentrate on. There are many uses for a traffic-pattern based searches, but 
right now we are limiting ourselves with the scenario where you would like to find similar
\footnote{Similarity is based on a combination of traffic-pattern and category. Relevancy is based on  
geospatial location radius} venues in the surrounding area. \\ \\
In addition, we have kicked off a parser that iterates through a 22 million lines of data, provided
by the research paper, into a sqlite3 database. The conversion took about a week, and the database
size is around 3 GB. For the sake of developments and efficiency, we might have to port over to a MySQL
database.

\subsection{Backend Approach}
There will be two databases for storing incoming tweets from Foursquare and for storing
location/venue data. \verb|Checkins| database will serve as a buffer between a script that is rate-limited
(to conform with Twitter API) and a tweet processor. In other words, the script will dump all the relevant 
tweets it can scrape into the \verb|Checkins| database, and on the other side, a processor will parse and
add additional information to the tweet, which is explained in the next part. \\ \\
If the \verb|Checkins| database is not empty, the processor reads every tweet, parses it, fetch
additional information, and adds it to the second database, \verb|locations|. Every location has an
unique ID from Twitter, and we use that to query the \verb|locations| database. If such location
exists, append the traffic and recompute the pattern. If not, start off parallel threads to
third-party APIs and fetch all the information there is on that location. For example, we ask Google
what the category, address, name, boundary box (for drawing on maps), and etc. If the location is a
restaurant, we can ask Yelp for reviews, ratings, hours of operations, and such. \\ \\

\section{Other Thoughts}
If you would like to fork our private Github repo, something can be arranged.

\end{document}

