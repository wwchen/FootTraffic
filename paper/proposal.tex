%
% Project proposal
% Guidelines can be found at: http://courses.cse.tamu.edu/caverlee/csce470/project.html
% 
\documentclass{article}
\renewcommand{\contentsname}{Table of Contents}
\newcommand{\PaperTitle}{Proposal}
\usepackage{hyperref}
%\usepackage{fullpage}

\begin{document}

%
% Title Page
\newcommand{\HRule}{\rule{\linewidth}{0.5mm}}
%
% Title Page
%
\begin{titlepage}
  \begin{center}
  % Title
  \textsc{\Large FootTraffic}\\[0.5cm]
  \HRule \\[0.4cm]
  {\huge \bfseries \PaperTitle}\\[0.2cm]
  \HRule \\[1cm]
  
  \large{Department of Computer Science \& Engineering} \\
  \large{Texas A\&M University} \\[1cm]

  % Authors and emails
  \begin{minipage}{0.4\textwidth}
    \begin{flushleft} \large
    \emph{Authors:}\\
    William \textsc{Chen} \\
    Eric \textsc{Wood}
    \end{flushleft}
  \end{minipage}
  \begin{minipage}{0.4\textwidth}
    \begin{flushright} \large
    \emph{Emails:} \\
    wchen16@cse.tamu.edu\\
    ebw0178@cse.tamu.edu
    \end{flushright}
  \end{minipage}
  \\[1cm]
  \large{\today}
  \vfill
  
  % Bottom of the page
  \end{center}
\end{titlepage}


%
% Table of Contents
%\tableofcontents
%\cleardoublepage

%
% Document
% What is exactly the function of your tool? That is, what will it do?
\section{Introduction}
We are seeing an increasing amount of location sharing from social services Foursquare, Facebook, Google Latitude, etc.,
and we want to harness these traffic to recommend venues to users based on the traffic patterns. Most location-based
search engines today rank results based on proximity, user rating, category, and popularity. We want to add another 
dimension to location-based searches, and that is ranking based on observed traffic similarities.
This ``\textit{temporal dynamics embedded in the checkins} from location sharing services''
\footnote{``\href{http://faculty.cs.tamu.edu/caverlee/pubs/cheng11cikm.pdf}{Toward Traffic-Driven Location-Based Web Search}''
  by Zhiyuan Cheng, James Caverlee, Krishna Y. Kamath, and Kyumin Lee; CIKM 2011},
approach is introduced by a research paper authored by Zhiyuan Cheng, James Caverlee, Krishna Y. Kamath,
and Kyumin Lee, and the fountain of FootTraffic search engine will be based on the ideas from the paper
\footnote{A copy of the paper can be found at: \url{http://faculty.cs.tamu.edu/caverlee/pubs/cheng11cikm.pdf}}
aforementioned.

% Why would we need such a tool and who would you expect to use it and benefit from it?
\section{Purpose}
% impact, need
Adding a traffic-driven dimension to location-based searches serves many purposes, and to illustrate, below
are some user-case scenarios:
\begin{enumerate}
\item Reviews are relevant to products, because their quality doesn't fluctuate, and availability (inventory) is
reflected real-time. For venues like restaurants, a raving review dated a year ago doesn't mean the restaurant
isn't infested with rats and ultimately closed down today. If we observe a plummet of checkins, we can
make our results more relevant and up to date. 
\item Suppose you are new to an area and you wish to see what bars in the local area airs the NFL football games.
You can review the traffic graph for the local bars and correlate the previous game days with the at-peak checkins
\item Tina and her friends are celebrating graduation, and they are looking for bars to hit late at night. What bars
are at-peak and what bars will be at-peak when they arrive? \footnote{Example is taken from the paper}
\item John has moved to a new area and would like to know which basketball courts have a similar traffic pattern
like the one at home, so he can meet new friends. \footnotemark[\value{footnote}]
\end{enumerate}
A traffic-driven location-based search can easily handle these kind of queries that a rating- and review-based
search cannot, whether it is by returning semantically correlated venues with similar traffic patterns, or by
having traffic patterns to make a decision.

% Would people care about the difference? How hard is it to build such a tool? What is the challenge?
\section{Design and Methods}
% How do you plan to build it?
\subsection{Design Specifications}
We plan on using the Ruby on Rails framework for several reasons. Not only will it allow us
to rapidly build the web application, it provides an overall architecture for us to build the data processing
on top of as well as an easy and somewhat transparent way to interact with our database. In addition, the MVC
architecture makes modulization easy, and that is a huge advantage for us. We plan on using
MariahDB, a fork of MySQL as our main database.

% What existing resources can you use?
\subsection{Development Resources}
For our search engine, we will be reusing the FourSquare checkin data collected from the research paper
which is 22 million checkins gathered from Twitter's public streaming, from October 2010 to January 2011. 
The data can be publicly downloaded at \href{http://infolab.tamu.edu/static/users/zhiyuan/icwsm_2011.zip}{infolab.tamu.edu}.
Using this dataset, we can extract a wealth of information, including location, time, category of the
venues, frequency, tags, reviews, features, photos, etc. Needless to say, this 3 GB of dataset is the
integral piece that drives our search engine.
\\ \\
In addition, in our preliminary brainstorm, we have decided to use Google Maps API to plot the different
venues in a geographical map, and Yelp API to show restaurant reviews. Other APIs will possibly used
as we add more modules to different venues and categories.
\\ \\
Last but not least, our class materials on optimization, indexing, evaluation, compression, and many more
will prove invaluable.
  
% How will you demonstrate the usefulness of your tool?
\subsection{Design Envision}
Users can submit queries just like they would to Google Maps, using specific locations or addresses,
or broad categories. We'll combine the Google results with information gathered from our database of
FourSquare checkins as well as any number of rating services. We will allow the user to sort their results
based on popularity for current or future dates, allowing them to find the location that fits their needs best.

\subsection{Design Challenges}
The main challenge we face, like any search engine, is relevancy and speed (optimization). In building FootTraffic,
we want to weigh our deciding factors (location, review, traffic, time, semantic relations, etc) at the optimal 
balance, and it may differ for the different types of queries. Also, keeping such a large dataset with millions
of locations and checkin data over the course of several years is not trivial, and querying it in a timely and
efficient manner makes it even more difficult. These problems are not new, however with the correct approach,
the amount of information is manageable.

% A brief timeline including a description of what exactly you will show us at the Project Checkpoint
\section{Management Plan}
\subsection{Roles}
To manage our time successfully, we have set a private Git repository on GitHub to manage our codebase. 
We plan to meet at least twice a week and continually have dialogues via IRC. We will split development
and documentation responsibilities equally.

\subsection{Schedule of tasks}
\begin{tabular}{|l|l|l|}
\hline
  & Tasks                             & Due Date   \\ \hline
1 & Submit proposal                   & 10/21/2011 \\
2 & Proposal feedback                 & 10/27/2011 \\
- & Complete Milestone Walk           & 10/30/2011 \\
- & Complete Milestone Jog            & 11/15/2011 \\
3 & Project checkpoint                & 11/15/2011 \\
4 & Two slides                        & 11/22/2011 \\
- & Complete Milestone Sprint         & 11/27/2011 \\
5 & Finish in-class presentation      & 11/29/2011 \\
- & Complete Milestone Run            & 12/10/2011 \\
6 & Finish executive summary + demo   & 12/12/2011 \\
\hline
\end{tabular}



\end{document}
