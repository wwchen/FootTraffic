%
% Project proposal
% Guidelines can be found at: http://courses.cse.tamu.edu/caverlee/csce470/project.html
% 
\documentclass{article}
\renewcommand{\contentsname}{Table of Contents}
\newcommand{\PaperTitle}{Proposal: A Traffic-Driven Location-Based Web Search Engine}
\usepackage{hyperref}
%\usepackage{fullpage}

\begin{document}

%
% Title Page
\newcommand{\HRule}{\rule{\linewidth}{0.5mm}}
%
% Title Page
%
\begin{titlepage}
  \begin{center}
  % Title
  \textsc{\Large FootTraffic}\\[0.5cm]
  \HRule \\[0.4cm]
  {\huge \bfseries \PaperTitle}\\[0.2cm]
  \HRule \\[1cm]
  
  \large{Department of Computer Science \& Engineering} \\
  \large{Texas A\&M University} \\[1cm]

  % Authors and emails
  \begin{minipage}{0.4\textwidth}
    \begin{flushleft} \large
    \emph{Authors:}\\
    William \textsc{Chen} \\
    Eric \textsc{Wood}
    \end{flushleft}
  \end{minipage}
  \begin{minipage}{0.4\textwidth}
    \begin{flushright} \large
    \emph{Emails:} \\
    wchen16@cse.tamu.edu\\
    ebw0178@cse.tamu.edu
    \end{flushright}
  \end{minipage}
  \\[1cm]
  \large{\today}
  \vfill
  
  % Bottom of the page
  \end{center}
\end{titlepage}


%
% Table of Contents
%\tableofcontents
%\cleardoublepage

%
% Document
% What is exactly the function of your tool? That is, what will it do?
\section{Introduction}
Location sharing services, such 
are beginning to take ground

% Why would we need such a tool and who would you expect to use it and benefit from it?
\section{Purpose}
% impact, need

% Does this kind of tools already exist? If similar tools exist, how is your tool different from them? 
% Would people care about the difference? How hard is it to build such a tool? What is the challenge?
\section{Related Work}

\section{Design and Methods}
% How do you plan to build it?
\subsection{Design Specifications}
Rails, database (mysql, sqlite3)
Modules, so Yelp reviews will only show up when the category of venues is restaurants

% What existing resources can you use?
\subsection{Development Resources}
For our search engine, we will be reusing the FourSquare checkin data from the research paper
\footnote{"Toward Traffic-Driven Location-Based Web Search" by Zhiyuan Cheng, James Caverlee, Krishna Y. Kamath, and Kyumin Lee, ACM 978-1-4503-0717-8/11/10},
which is 22 million checkins gathered from Twitter's public streaming, from October 2010 to January 2011. 
The data can be publicly downloaded at \href{http://infolab.tamu.edu/static/users/zhiyuan/icwsm_2011.zip}{infolab.tamu.edu}.
Using this dataset, we can extract a wealth of information, including location, time, category of the venues, frequency, tags, reviews, features, photos, etc.
Needless to say, this 3 GB of dataset is the integral piece that drives our search engine.
\\ \\
In addition, in our preliminary brainstorm, we have decided to use Google Maps API to plot the different venues in a geographical map, and
Yelp API to show restaurant reviews. Other APIs will possibly used as we add more modules to different venues and categories.

% How will you demonstrate the usefulness of your tool?
\subsection{Broader Impact and Needs}

% A brief timeline including a description of what exactly you will show us at the Project Checkpoint
\section{Management Plan}
\subsection{Roles}
To manage our time successfully, we have set a private Git repository on GitHub to manage our codebase. 
We plan to meet twice a week and continually have dialogues via IRC. We will split development and documentation
responsibilities equally.

\subsection{Schedule of tasks}
\begin{tabular}{|l|l|l|}
\hline
  & Tasks                             & Due Date   \\ \hline
- & Submit proposal                   & 10/21/2011 \\
- & Proposal feedback                 & 10/27/2011 \\
> & Complete Milestone Jog            & 10/30/2011 \\
> & Complete Milestone Hurdling       & 11/15/2011 \\
- & Project checkpoint                & 11/15/2011 \\
- & Two slides                        & 11/22/2011 \\
> & Complete Milestone Sprint         & 11/27/2011 \\
- & Finish in-class presentation      & 11/29/2011 \\
> & Complete Milestone Run            & 12/10/2011 \\
- & Finish executive summary + demo   & 12/12/2011 \\
\hline
\end{tabular}



\end{document}
