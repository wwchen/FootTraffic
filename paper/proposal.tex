%
% Project proposal
% Guidelines can be found at: http://courses.cse.tamu.edu/caverlee/csce470/project.html
% 
\documentclass{article}
\renewcommand{\contentsname}{Table of Contents}
\newcommand{\PaperTitle}{Proposal: A Traffic-Driven Location-Based Web Search Engine}
\usepackage{hyperref}
%\usepackage{fullpage}

\begin{document}

%
% Title Page
\newcommand{\HRule}{\rule{\linewidth}{0.5mm}}
%
% Title Page
%
\begin{titlepage}
  \begin{center}
  % Title
  \textsc{\Large FootTraffic}\\[0.5cm]
  \HRule \\[0.4cm]
  {\huge \bfseries \PaperTitle}\\[0.2cm]
  \HRule \\[1cm]
  
  \large{Department of Computer Science \& Engineering} \\
  \large{Texas A\&M University} \\[1cm]

  % Authors and emails
  \begin{minipage}{0.4\textwidth}
    \begin{flushleft} \large
    \emph{Authors:}\\
    William \textsc{Chen} \\
    Eric \textsc{Wood}
    \end{flushleft}
  \end{minipage}
  \begin{minipage}{0.4\textwidth}
    \begin{flushright} \large
    \emph{Emails:} \\
    wchen16@cse.tamu.edu\\
    ebw0178@cse.tamu.edu
    \end{flushright}
  \end{minipage}
  \\[1cm]
  \large{\today}
  \vfill
  
  % Bottom of the page
  \end{center}
\end{titlepage}


%
% Table of Contents
%\tableofcontents
%\cleardoublepage

%
% Document
% What is exactly the function of your tool? That is, what will it do?
\section{Introduction}
FootTraffic is a location recommendation engine that, unlike existing services, takes into account the
location's popularity based on checkin information courtesy of FourSquare.

% Why would we need such a tool and who would you expect to use it and benefit from it?
\section{Purpose}
% impact, need
People love mapping and rating services. Looking for a specific genre of restaurant in your area with
good prices and reviews? There's plenty of tools out there to help you out!
\\ \\
While this is useful information, you don't quite get the full picture. Does this restaurant have notoriously
long waits on Friday nights? A reviewer may point that out, but you can't always count on it, and even
then it isn't taken into account.
\\ \\
FootTraffic, however, doesn't leave this information out. We take data from FourSquare checkins, allowing
us to guage popularity during times of the day and year. Stuck waiting for a table at a TexMex restaurant
and wishing you could find another one close-by that is still rated well and known to be less busy at
that time of day? We can help you with that! Looking for the local hotspots at night? We can find them for you.  
\\ \\
Users can submit queries just like they would to Google Maps, using specific locations or addresses,
or broad categories. We'll combine the Google results with information gathered from our database of
FourSquare checkins as well as any number of rating services. We'll allow the user to sort their results
based on popularity for current or future dates, allowing them to find the location that fits their needs best.
\\ \\
The main challenge we face in building FootTraffic is figuring out how best to weight our location data
to determine overall popularity. Keeping such a large dataset with millions of locations and checkin
data over the course of several years is not trivial, and querying it in a timely manner makes it more
difficult. These problems are not new, however, and with the correct approach the amount of information is manageable.
\\ \\
We have to pull in at least three sources of information: GoogleMaps data to determine the location (as
well as review data), Yelp for restaurant ratings (easily swapped out or augmented with other rating
websites), as well as our database of FourSquare checkins. Returning relevant results to our users means
processing this data and weighting our results in a manner that best represents their desires. It will
take some trial and error to determine this.

% Would people care about the difference? How hard is it to build such a tool? What is the challenge?
\section{Related Work}

\section{Design and Methods}
% How do you plan to build it?
\subsection{Design Specifications}
Since this a web application, we plan on using the Ruby on Rails framework. Not only will it allow us
to rapidly build the web application, it provides an overall architecture for us to build the data processing
on top of as well as an easy and somewhat transparent way to interact with our database. We plan on using
MariahDB, a fork of MySQL as our main database.

% What existing resources can you use?
\subsection{Development Resources}
For our search engine, we will be reusing the FourSquare checkin data from the research paper
\footnote{"Toward Traffic-Driven Location-Based Web Search" by Zhiyuan Cheng, James Caverlee, Krishna Y. Kamath, and Kyumin Lee, ACM 978-1-4503-0717-8/11/10},
which is 22 million checkins gathered from Twitter's public streaming, from October 2010 to January 2011. 
The data can be publicly downloaded at \href{http://infolab.tamu.edu/static/users/zhiyuan/icwsm_2011.zip}{infolab.tamu.edu}.
Using this dataset, we can extract a wealth of information, including location, time, category of the
venues, frequency, tags, reviews, features, photos, etc. Needless to say, this 3 GB of dataset is the
integral piece that drives our search engine.
\\ \\
In addition, in our preliminary brainstorm, we have decided to use Google Maps API to plot the different
venues in a geographical map, and Yelp API to show restaurant reviews. Other APIs will possibly used
as we add more modules to different venues and categories.

% How will you demonstrate the usefulness of your tool?
\subsection{Broader Impact and Needs}

% A brief timeline including a description of what exactly you will show us at the Project Checkpoint
\section{Management Plan}
\subsection{Roles}
To manage our time successfully, we have set a private Git repository on GitHub to manage our codebase. 
We plan to meet at least twice a week and continually have dialogues via IRC. We will split development
and documentation responsibilites equally.

\subsection{Schedule of tasks}
\begin{tabular}{|l|l|l|}
\hline
  & Tasks                             & Due Date   \\ \hline
- & Submit proposal                   & 10/21/2011 \\
- & Proposal feedback                 & 10/27/2011 \\
> & Complete Milestone Jog            & 10/30/2011 \\
> & Complete Milestone Hurdling       & 11/15/2011 \\
- & Project checkpoint                & 11/15/2011 \\
- & Two slides                        & 11/22/2011 \\
> & Complete Milestone Sprint         & 11/27/2011 \\
- & Finish in-class presentation      & 11/29/2011 \\
> & Complete Milestone Run            & 12/10/2011 \\
- & Finish executive summary + demo   & 12/12/2011 \\
\hline
\end{tabular}



\end{document}
